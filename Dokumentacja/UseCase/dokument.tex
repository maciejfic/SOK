\documentclass[12pt]{article}
    % ------------------------------------------------------------------------
% PAKIETY
% ------------------------------------------------------------------------

%r�ne pakiety matematyczne, warto przejrze� dokumentacj�, musz� by� powy�ej ustawie� j�zykowych.
\usepackage{mathrsfs}   %R�ne symbole matematyczne opisane w katalogu ~\doc\latex\comprehensive. Zamienia \mathcal{L} ze zwyk�ego L na L-transformat�.
\usepackage{eucal}      %R�ne symbole matematyczne.
\usepackage{amssymb}    %R�ne symbole matematyczne.
\usepackage{amsmath}    %Dodatkowe funkcje matematyczne, np. polecenie \dfac{}{} skladajace ulamek w trybie wystawionym (por�wnaj $\dfrac{1}{2}$, a $\frac{1}{2}$).

%j�zyk polski i klawiatura
\usepackage[polish]{babel}
\usepackage[OT4]{polski}
\usepackage[cp1250]{inputenc}                       %Strona kodowa polskich znak�w.

%obs�uga pdf'a
\usepackage[pdftex,usenames,dvipsnames]{color}      %Obs�uga kolor�w. Opcje usenames i dvipsnames wprowadzaj� dodatkowe nazwy kolorow.
\usepackage[pdftex,pagebackref=false,draft=false,pdfpagelabels=false,colorlinks=true,urlcolor=blue,linkcolor=red,citecolor=green,pdfstartview=FitH,pdfstartpage=1,pdfpagemode=UseOutlines,bookmarks=true,bookmarksopen=true,bookmarksopenlevel=2,bookmarksnumbered=true,pdfauthor={Marcin Szewczyk},pdftitle={Doktorat},pdfsubject={Praca doktorska},pdfkeywords={transient recovery voltage trv},unicode=true]{hyperref}   %Opcja pagebackref=true dotyczy bibliografii: pokazuje w spisie literatury numery stron, na kt�rych odwo�ano si� do danej pozycji.

%bibliografia
\usepackage[numbers,sort&compress]{natbib}  %Porz�dkuje zawarto�� odno�nik�w do literatury, np. [2-4,6]. Musi by� pod pdf'em, a styl bibliogfafii musi mie� nazw� z dodatkiem 'nat', np. \bibliographystyle{unsrtnat} (w kolejno�ci cytowania).
\usepackage{hypernat}                       %Potrzebna pakietowi natbib do wspolpracy z pakietem hyperref (wazna kolejnosc: 1. hyperref, 2. natbib, 3. hypernat).

%grafika i geometria strony
\usepackage{extsizes}           %Dostepne inne rozmiary czcionek, np. 14 w poleceniu: \documentclass[14pt]{article}.
\usepackage[final]{graphicx}
\usepackage[a4paper,left=3.5cm,right=2.5cm,top=2.5cm,bottom=2.5cm]{geometry}

%strona tytu�owa
\usepackage{strona_tytulowa}

%inne
\usepackage[hide]{todo}                     %Wprowadza polecenie \todo{tre��}. Opcje pakietu: hide/show. Polecenie \todos ma byc na koncu dokumentu, wszystkie \todo{} po \todos sa ignorowane.
\usepackage[basic,physics]{circ}            %Wprowadza �rodowisko circuit do rysowania obwod�w elektrycznych. Musi byc poni�ej pakietow j�zykowych.
\usepackage[sf,bf,outermarks]{titlesec}     %Troszczy si� o wygl�d tytu��w rozdzia��w (section, subsection, ...). sf oznacza czcionk� sans serif (typu arial), bf -- bold. U mnie: oddzielna linia dla naglowku paragraph. Patrz tez: tocloft -- lepiej robi format spisu tresci.
\usepackage{tocloft}                        %Troszczy si� o format spisu trsci.
\usepackage{expdlist}    %Zmienia definicj� �rodowiska description, daje wi�ksze mo�liwo�ci wp�ywu na wygl�d listy.
\usepackage{flafter}     %Wprowadza parametr [tb] do polecenia \suppressfloats[t] (polecenie to powoduje nie umieszczanie rysunkow, tabel itp. na stronach, na ktorych jest to polecenie (np. moze byc to stroma z tytulem rozdzialu, ktory chcemy zeby byl u samej gory, a nie np. pod rysunkiem)).
\usepackage{array}       %�adniej drukuje tabelki (np. daje wiecej miejsca w komorkach -- nie s� tak �cie�nione, jak bez tego pakietu).
\usepackage{listings}    %Listingi programow.
\usepackage[format=hang,labelsep=period,labelfont={bf,small},textfont=small]{caption}   %Formatuje podpisy pod rysunkami i tabelami. Parametr 'hang' powoduje wci�cie kolejnych linii podpisu na szerokosc nazwy podpisu, np. 'Rysunek 1.'.
\usepackage{appendix}    %Troszczy si� o za��czniki.
\usepackage{floatflt}    %Troszczy si� o oblewanie rysunkow tekstem.
\usepackage{here}        %Wprowadza dodtkowy parametr umiejscowienia rysunk�w, tabel, itp.: H (du�e). Umiejscawia obiekty ruchome dokladnie tam gdzie s� w kodzie �r�d�owym dokumentu.
\usepackage{makeidx}     %Troszczy si� o indeks (skorowidz).

%nieu�ywane, ale potencjalnie przydatne
%\usepackage{sectsty}           %Formatuje nag��wki, np. �eby by�y kolorowe -- polecenie: \allsectionsfont{\color{Blue}}.
%\usepackage{version}           %Wersje dokumentu.
%\usepackage{fancyhdr}          %Dodaje naglowki jakie si� chce.
%\usepackage{antyktor}          %Sk�ada dokument przy u�yciu Antykwy Toru�skiej.
%\usepackage{antpolt}           %Sk�ada dokument przy u�yciu Antykwy P�tawskiego.
%\usepackage[left]{showlabels}  %Pokazuje etykiety, ale kiepsko bo nie mieszcza sie na marginesie (mo�na od biedy powi�kszy� margines w pakiecie geometry powy�ej). Nie mo�e by� na g�rze (pakiet).

    % ------------------------------------------------------------------------
% USTAWIENIA
% ------------------------------------------------------------------------

% ------------------------------------------------------------------------
%   Kropki po numerach sekcji, podsekcji, itd.
%   Np. 1.2. Tytuł podrozdziału
% ------------------------------------------------------------------------
\makeatletter
    \def\numberline#1{\hb@xt@\@tempdima{#1.\hfil}}                      %kropki w spisie treści
    \renewcommand*\@seccntformat[1]{\csname the#1\endcsname.\enspace}   %kropki w treści dokumentu
\makeatother

% ------------------------------------------------------------------------
%   Numeracja równań, rysunków i tabel
%   Np.: (1.2), gdzie:
%   1 - numer sekcji, 2 - numer równania, rysunku, tabeli
%   Uwaga ogólna: o otoczeniu figure ma być najpierw \caption{}, potem \label{}, inaczej odnośnik nie działa!
% ------------------------------------------------------------------------
\makeatletter
    \@addtoreset{equation}{section} %resetuje licznik po rozpoczęciu nowej sekcji
    \renewcommand{\theequation}{{\thesection}.\@arabic\c@equation} %dodaje kropkę

    \@addtoreset{figure}{section}
    \renewcommand{\thefigure}{{\thesection}.\@arabic\c@figure}

    \@addtoreset{table}{section}
    \renewcommand{\thetable}{{\thesection}.\@arabic\c@table}
\makeatother

% ------------------------------------------------------------------------
% Tablica
% ------------------------------------------------------------------------
\newenvironment{tablica}[3]
{
    \begin{table}[!tb]
    \centering
    \caption[#1]{#2}
    \vskip 9pt
    #3
}{
    \end{table}
}

% ------------------------------------------------------------------------
% Dostosowanie wyglądu pozycji listy \todos, np. zamiast 'p.' jest 'str.'
% ------------------------------------------------------------------------
\renewcommand{\todoitem}[2]{%
    \item \label{todo:\thetodo}%
    \ifx#1\todomark%
        \else\textbf{#1 }%
    \fi%
    (str.~\pageref{todopage:\thetodo})\ #2}
\renewcommand{\todoname}{Do zrobienia...}
\renewcommand{\todomark}{~uzupełnić}

% ------------------------------------------------------------------------
% Definicje
% ------------------------------------------------------------------------
\def\nonumsection#1{%
    \section*{#1}%
    \addcontentsline{toc}{section}{#1}%
    }
\def\nonumsubsection#1{%
    \subsection*{#1}%
    \addcontentsline{toc}{subsection}{#1}%
    }
\reversemarginpar %umieszcza notki po lewej stronie, czyli tam gdzie jest więcej miejsca
\def\notka#1{%
    \marginpar{\footnotesize{#1}}%
    }
\def\mathcal#1{%
    \mathscr{#1}%
    }
\newcommand{\atp}{ATP/EMTP} % Inaczej: \def\atp{ATP/EMTP}

% ------------------------------------------------------------------------
% Inne
% ------------------------------------------------------------------------
\frenchspacing                      %nie pamiętam co to jest, ale używam
%\flushbottom                       %nie pamiętam co to jest, ale nie używam
%\raggedbottom                      %nie pamiętam co to jest, ale nie używam
\hyphenation{ATP/-EMTP}             %dzielenie wyrazu w żądanym miejscu
\setlength{\parskip}{1pt}           %odstęp pomiędzy akapitami
\linespread{1.2}                    %odstęp pomiędzy liniami (interlinia)
\setcounter{tocdepth}{4}            %uwzględnianie w spisie treści czterech poziomów sekcji
\setcounter{secnumdepth}{4}         %numerowanie do czwartego poziomu sekcji włącznie
\titleformat{\paragraph}[hang]      %wygląd nagłówków
{\normalfont\sffamily\bfseries}{\theparagraph}{1em}{}
%\definecolor{niebieski}{rgb}{0.0,0.0,0.5}

    % !TEX program = pdflatex
    \title{Specyfikacja ról i przypadków użycia dla Systemu Obsługi Konferencji instytutu Sterowania i Elektroniki Przemysłowej Politechniki Warszawskiej }
    \author{Piotr Błądek}

    %polecenia zdefiniowane w pakiecie strona_tytulowa.sty
    \uczelnia{POLITECHNIKA WARSZAWSKA}
    \instytut{Instytut Sterowania i Elektroniki\\ Przemysłowej Politechniki Warszawskiej}
    \promotor{dr inż. Ryszarda Łagody}
    \praca{Projekt Zespołowy}
    \rok{2015}
    %\draft  %odkomentarzuj tę linię, jeśli wydruk ma być draftem -- wprowadza informację o drafcie na stronie tytułowej
    %koniec poleceń zdefiniowanych przez pakiet strona_tytulowa.sty

    %skorowidz - nie używam
    %\makeindex
    %koniec skorowidza

\begin{document}
    \renewcommand{\figurename}{Rys.}    %musi byc pod \begin{document}, bo w~tym miejscu pakiet 'babel' narzuca swoje ustawienia
    \renewcommand{\tablename}{Tab.}     %j.w.
    %\pagenumbering{roman}               %numeracja stron: rzymska
    \thispagestyle{empty}               %na tej stronie: brak numeru
    \stronatytulowa                     %strona tytułowa tworzona przez pakiet strona_tytulowa.tex

    %podziękowania
    %\newpage
    %~   %potrzebne dla \vfill
    %\vfill
    %{\sffamily
    %\begin{flushright}
      %  \begin{tabular}{l}
       % Chciałbym w~tym miejscu podziękować:\\
       % \\
       % Kolegom i koleżankom z Klubu\\
        %-- za udane wakacje, które przyczyniły się\\
        %do powstania tej pracy\\
        %\\
        %Bosmanowi ze Sztynortu\\
        %-- za naprawienie silnika, itp.\\
        %\end{tabular}
    %\end{flushright}
    %}
    %\vskip0.5in
    %\thispagestyle{empty}
    
    
    
    %\newpage
    %koniec podziękowań

    %formatowanie spisu treści i~nagłówków
    \renewcommand{\cftbeforesecskip}{8pt}
    \renewcommand{\cftsecafterpnum}{\vskip 8pt}
    \renewcommand{\cftparskip}{3pt}
    \renewcommand{\cfttoctitlefont}{\Large\bfseries\sffamily}
    \renewcommand{\cftsecfont}{\bfseries\sffamily}
    \renewcommand{\cftsubsecfont}{\sffamily}
    \renewcommand{\cftsubsubsecfont}{\sffamily}
    \renewcommand{\cftparafont}{\sffamily}
    %koniec formatowania spisu treści i nagłówków

    \tableofcontents    %spis treści
    %\newpage

    %\input{wykaz_skrotow.tex}
    %\newpage

    %spis rysunków i~tablic
    %format spisu treści dotyczy tylko tych spisów, które są poniżej
    \hypersetup{linkcolor=black}
    \renewcommand{\cftparskip}{3pt}
    \renewcommand{\cftloftitlefont}{\Large\bfseries\sffamily}
    \listoffigures
    \addcontentsline{toc}{section}{Spis rysunków}
    %\addcontentsline{toc}{section}{Spis tablic}
    \newpage
    %powrót do czerwonego koloru dla dalszych odnośników (spis literatury, przy włączonym pokazywaniu numerów stron, do których odnoszą się poszczególne pozycje)
    \hypersetup{linkcolor=black}
    %koniec spisu rysunków i~tablic

    %przeniesienie wartości licznika stron na strony numerowane innym stylem (arabic)
    \newcounter{licznikStron}
    \setcounter{licznikStron}{\value{page}}
    \pagenumbering{arabic}
    \setcounter{page}{\value{licznikStron}}
    %koniec przeniesienia...

    %treść główna dokumentu
    \section{Cel dokumentu}
\suppressfloats[t]

Dokument ma na celu definicję
wszystkich wymagań, jakie ma spełniać system.

\section{Ostateczny opis wymagań}

\subsection{Wstęp}
System ma na celu ułatwienie i zautomatyzowanie procesu organizacji
dorocznych konferencji instytutu. Podczas takiego wydarzenia pracownicy 
wraz z członkami rodzin wyjeżdżają na kilka dni. Pracownicy w tym czasie
uczestniczą w konferencjach przez nich przygotowanych prezentując własne
artykuły wcześniej zrecenzowane przez osoby spoza instytutu. Rodziny 
uczestniczą w wycieczkach. \par
W przygotowaniach do konferencji będą uczestniczyli pracownicy,
oraz członkowie ich rodzin wykorzystując System Obsługi Konferencji. Każda
zainteresowana osoba będzie miała do niego dostęp i będzie brała udział
w interesujących ją procesach przygotowujących konferencję.
\subsection{Definicja wymagań funkcjonalnych}
\subsubsection{Administrator}
Administrator, jako osoba mająca absolutną kontrolę nad systemem, nad którą
spoczywa odpowiedzialność za całościowe przygotowanie konferencji może 
zarządzać innymi użytkownikami i konfiguracją systemu. \par
Jedną z podstawowych ról administratora jest umożliwienie dostępu do
Systemu pozostałym zainteresowanym. W tym celu Administrator ma możliwość 
utworzyć konto odpowiedniej osobie naj jej wniosek lub z własnej 
inicjatywy. Wyróżniamy 4 rodzaje kont użytkowników:
\begin{description}
    \item[Konto Administratora] \hfill \\
        konto z pełnym dostępem do wszystkich funkcji systemu.
        Do poprawnego działania aplikacji musi być aktywne co
        najmniej jedno konto Administratora. To on przydziela uprawnienia,
        zadania, role i dostępy pozostałym członkom.

    \item[Konto Pracownika] \hfill \\
        Każdy pracownik instytutu powinien posiadać to konto, 
        o ile chce brać aktywny udział w Konferencji. Podstawowymi 
        własnościami takiego konta, jest możliwość zgłoszenia swojego 
        artykułu do recenzji i późniejszego poprowadzenia prelekcji.
        Do konta pracowniczego dołączane są konta członków rodziny 
        pracownika.

    \item[Konto Członka rodziny] \hfill \\
        Może należeć do jednej lub większej ilości osób. Musi być 
        przypisane do dokładnie jednego pracownika. Nie daje dostępu do
        naukowo-dydaktycznej części systemu.

    \item[Dostęp Recenzenta] \hfill \\
        Nie jest kontem w pełnym tego słowa znaczeniu, ponieważ
        ma za zadanie dać możliwość uczestnictwa recenzentów w procesie
        oceniania artykułu. System powinien działać w sposób przezroczysty
        dla recenzenta, jednocześnie ukrywając tożsamość autora artykułu
        i recenzenta przed nimi samymi.

\end{description}
Administrator powinien mieć też możliwość wyświetlenia listy wszystkich 
kont w systemie i ich uprawnień. Ponadto dla każde wyświetlane konto
może edytować. Oznacza to, że może edytować dane opisujące to konto,
ustawić nowe hasło dla użytkownika, a także usunąć konto, lub je 
zamknąć. \par
Administrator w procesie oceny artykuły ma możliwość oddelegować 
zarządzanie wycieczkami do innego użytkownika, oraz zaakceptować
program poszczególnej wycieczki. \par
W procesie oceny artykułu, administrator może ocenić postęp zaawansowania
postępów oceny, może zaingerować w sam proces.\par
\subsubsection{Ocena artykułu}
Artykuł jest dokumentem napisanym przez pracownika, na temat którego
w czasie konferencji ma zostać wygłoszona prelekcja. Powinien on zostać
wcześniej zrecenzowany przez osobę zaznajomioną z tematem, o którym
artykuł traktuje. Jednocześnie recenzent nie powinien znać tożsamości
autora artykułu i vice versa. Z tego powodu Recenzentami są najczęściej
osoby spoza instytutu. Proces recenzji artykułu polega na naprzemiennej
wymianie uwag i sugestii pomiędzy recenzentem i autorem, nanoszeniu
poprawek do momentu zaakceptowania artykułu przez recenzenta. 
\subsubsection{Pracownik}
Konto pracownicze daje dostęp do zasobów systemu ukierunkowanych na
przygotowanie części merytorycznej konferencji. Każdy pracownik może 
zgłosić swój artykuł, głosować na inne, oraz mieć podgląd na ranking 
najwyżej ocenionych artykułów, które zostały akceptowane. \par
Pracownik, jako osoba zainteresowana dyskusjami dydaktycznymi może 
zgłaszać propozycje takich tematów, komentować inne propozycje.
W kwestii wycieczek i atrakcji turystycznych Pracownik ma identyczne 
uprawnienia jak Członek Rodziny.
\subsubsection{Członek Rodziny}
Członek Rodziny jest osobą towarzyszącą Pracownikowi podczas Konferencji.
Pojedyncze konto Członka Rodziny może należeć jednocześnie do jednej lub
kilku osób (np. matki i małego dziecka). Członek Rodziny ma pełen dostęp
do części systemu odpowiedzialnej za organizację wycieczek i innych
atrakcji poza naukowych. Może on wyświetlić listę wycieczek, zaproponować
nową, wyświetlić jej szczegóły, zapisać się na atrakcję turystyczną,
komentować. Dodatkowo może poznać ofertę gastronomiczną miejsc odwiedzanych
podczas konferencji, wyświetlić harmonogram całej imprezy,
oraz odzyskać zapomniane lub utracone hasło.

\section{Opis ewolucji systemu}
Kolejnym etapem życia systemu po przeanalizowaniu wymagań systemu, jest
jego implementacja. System Obsługi Konferencji ma służyć zakładowi przez
wiele lat i umożliwiać rozszerzanie jego funkcjonalności zależnie od 
potrzeb.


\section{Historia Zmian}

\begin{tabularx}{\textwidth}{X|l|X}
\hline
\textbf{Data zmiany} & \textbf{Kto zmienił} & \textbf{Co zostało zmienione} \\ \hline
16 czerwca 2015          & Wojciech Decker & Utworzenie dokumentu.         \\ \hline
\end{tabularx}

    \newpage
    %koniec treści głównej dokumentu

    %bibliografia
    %GATHER{bibliografia.bib}                       %polecenie programu WinEdt, włącza plik do drzewa 'projektu'
    %\bibliographystyle{unsrtnat}                    %styl bibliografii: w kolejności cytowania, zrozumiały dla pakietu hyperref.sty
    %\addcontentsline{toc}{section}{Literatura}      %musi być wyżej niż \bibliography{bibliografia}, żeby w~spisie treści był właściwy numer strony
    %\bibliography{bibliografia}
    %koniec bibliografii

    %załączniki
    %\newpage
    %\appendix
    %\renewcommand{\appendixtocname}{Załączniki}
    %\renewcommand{\appendixpagename}{~\vspace{8cm} \begin{center}Załączniki\end{center}\newpage}
    %\thispagestyle{empty}
    %\addappheadtotoc
    %\appendixpage
    %\input{zalacznik.tex}
    %koniec załączników

    %skorowidz - nie używam
%    \index{Napis}
%    \index{Napis!Podnapis}
%    \printindex
    %koniec skorowidza

    %lista rzeczy do zrobienia: wypisuje na końcu dokumentu, patrz: pakiet todo.sty
    \todos
    %koniec listy rzeczy do zrobienia
\end{document}
