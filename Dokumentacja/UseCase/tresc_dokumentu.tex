\section{Cel dokumentu}
\suppressfloats[t]

Niniejszy dokument powstał w celu przedstawienia wszystkich poziomów uprawnień które znajdą się w Systemie Obsługi Konferencji (SOK), oraz akcji (przypadków użycia) jakie użytkownik o danym zakresie uprawnień może wykonać w systemie. 

\section{Poziomy uprawnień}

W systemie, zgodnie z wstępnymi ustaleniami, zostaną zaimplementowane 3 poziomy uprawnień. \newline
Pierwszą grupę (najprawdopodobniej jednoosobową) stanowili będą użytkownicy o poziomie uprawnień ADMINISTRATOR SYSTEMU, jest to grupa o prawach nadrzędnych, będzie widziała co się dzieje w całym systemie. Do członków tej grupy należało będzie tworzenie konferencji, zarządzanie nią, dodawanie kont pracowników i inne akcje, o czym później. \newline
Druga grupa to użytkownicy o poziomie uprawnień PRACOWNIK, pracownik otrzyma swoje konto od administratora systemu po utrzednim zgłoszeniu takiej chęci. Pracownik będzie mógł dodawać konto swojej rodzinie, dodać temat do dyskusji do kalendarza konferencji itd. \newline
Ostatnia, trzecia grupa to RODZINA PRACOWNIKA, jak sama nazwa wskazuje będzie to konto które będzie przysługiwało kilku osobom i tu krótkie wyjaśnienie. Pracownik najczęściej do konta rodzinnego doda żonę lub męża i ewentualnie dzieci które w większości nie będą najprawdopodobniej umiały posługiwać się systemem z powodu swojego młodzięczego wieku, dlatego do konta rodzina pracownika będzie można przypisać więcej niż jedną osobę, oczywiście pracownik będzie mógł dodać kilka kont typu rodzina pracownika, co może być przydatne jeżeli na przykład poza żoną i dziećmi będzie chciał zaprosić kolegę, lub siostrę na wyjazd. Czyli podsumowując, do jednego konta typu rodzina pracownika będzie można dodać kilka fizycznych osób. Konto to będzie miało dostęp tylko do części krajoznawczo turystycznej, innych nie będzie widział.

\section{Przypadki użycia}

Słowo wyjaśnienia, przypadki użycia to technika stosowana w inżynierii oprogramowania w celu opisania wymagań tworzonego systemu informatycznego. Przypadek użycia przedstawia interakcję pomiędzy aktorem (użytkownikiem systemu), który inicjuje zdarzenie oraz samym systemem.\newline
Poniżej zostaną przedstawione wstępnie przygotowane diagramy przypadków użycia użytkowników o konkretnych uprawnieniach.

\subsection{Administrator Systemu}
\subsection{Pracownik}
\subsection{Rodzina pracownika}

\section{Historia Zmian}

\begin{tabularx}{\textwidth}{X|l|X}
\hline
\textbf{Data zmiany} & \textbf{Kto zmienił} & \textbf{Co zostało zmienione} \\ \hline
11 mar 2015          & Błądek Piotr         & Utworzenie dokumentu          \\ \hline
11 mar 2015          &Błądek Piotr          &  Dodanie historii zmian            \\ \hline
\end{tabularx}