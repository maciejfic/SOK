\section{Cel dokumentu}
\suppressfloats[t]

Dokument ma na celu specyfikację wymagań funkcjonalnych dla Systemu Obsługi Konferencji (SOK). Wymagania przedstawione są za pomocą tzw. Historii Użytkownika.


\section{Historie Użytkownika}

\subsection{Administrator}

\begin{description}
  \item[Jako administrator chcę:] \hfill \\
  \begin{itemize}
    \item zobaczyć listę kont w systemie.\\
    \item założyć konto w systemie.\\
    \item edytować konto w systemie.\\
    \item wygenerować hasło dla użytkownika.\\
    \item zamknąć konto użytkownika.\\
    \item usunąć konto użytkownika.\\
  \end{itemize}

  \item[Jako administrator chcę:] \hfill \\
  \begin{itemize}
    \item wyznaczyć zarządcę wycieczek.\\
    \item zaakceptować program wycieczki.\\
  \end{itemize}

  \item[Jako administrator chcę:] \hfill \\
  \begin{itemize}
    \item ocenić postęp w procesie oceny artykułu pracownika.\\
    \item wziąć udział w procesie oceny artykułu pracownika.\\
  \end{itemize}

\end{description}




\subsection{Zgłaszający artykuł}

\begin{description}
  \item[Jako zgłaszający artykuł chcę:] \hfill \\
  \begin{itemize}
    \item znać uwagi recenzenta który zwrócił mi go do poprawy tak abym mógł wyprowadzić poprawki.\\
    \item mieć możliwość odpowiedzi na przedstawione mi uwagi jeżeli artykuł jest zwracany mi do poprawki.\\
    \item i oceniający artykuł chcemy mieć możliwość dwukierunkowej komunikacji odnośne danego artykułu.
  \end{itemize}
\end{description}




\subsection{Recenzent}

\begin{description}
  \item[Jako recenzent chcę:] \hfill \\
  \begin{itemize}
    \item ocenić artykuł.\\
    \item zaakceptować artykuł.\\
  \end{itemize}
\end{description}




\subsection{Pracownik}

\begin{description}
  \item[Jako pracownik chcę:] \hfill \\
  \begin{itemize}
    \item zgłosić swój artykuł do udziału w konferencji. \\
    \item zobaczyć listę zaakceptowanych artykułów.\\
    \item anonimowo głosować na najciekawsze artykuły.\\
    \item zobaczyć ranking najciekawszych artykułów.\\
  \end{itemize}

  \item[Jako pracownik chcę:] \hfill \\
  \begin{itemize}
    \item założyć konto rodzinie. \\
    \item poprosić o wygenerowanie dla mnie nowego hasła ponieważ utraciłem obecne.\\
  \end{itemize}

  \item[Jako pracownik chcę:] \hfill \\
  \begin{itemize}
    \item zaproponować temat dyskusji dydaktycznej.\\
    \item zobaczyć zgłoszone tematy dyskusji dydaktycznych.\\
    \item zamieścić komentarz odnośnie tematu zgłoszonego do dyskusji dydaktycznej.\\
  \end{itemize}

  \item[Jako pracownik chcę:] \hfill \\
  \begin{itemize}
    \item łatwo uzyskać podstawowe informacje organizacyjne o bieżącej konferencji.\\
    \item zobaczyć listę wycieczek.\\
    \item zobaczyć szczegóły wycieczki.\\
    \item zamieścić komentarz odnosnie wycieczki.\\
    \item zobaczyć listę lokalnych atrakcjii w miejscu konferencji.\\
    \item zapisać się na udział w lokalnej atrakcji.\\
    \item zobaczyć spis placówek medycznych w miejscu konferencji.\\
    \item zapoznać się z ofertą gastronomiczną.\\
  \end{itemize}

\end{description}




\subsection{Członek rodziny}

\begin{description}
  \item[Jako członek rodziny chcę:] \hfill \\
  \begin{itemize}
    \item zobaczyć listę wycieczek. \\
    \item zobaczyć szczegóły wycieczki.\\
    \item zamieścić komentarz odnosnie wycieczki.\\
    \item zapisać się na wycieczkę.\\
    \item zobaczyć listę lokalnych atrakcji w miejscu konferencji.\\
    \item zapisać się na udział w lokalnej atrakcji.\\
    \item zobaczyć opis formalny konferencji.\\
    \item zobaczyć harmonogram konferencji.\\
    \item zapoznać się z ofertą gastronomiczną.\\
    \item poprosić o wygenerowanie dla mnie nowego hasła ponieważ utraciłem obecne.\\
  \end{itemize}

\end{description}

\section{Historia Zmian}

\begin{tabularx}{\textwidth}{X|l|X}
\hline
\textbf{Data zmiany} & \textbf{Kto zmienił} & \textbf{Co zostało zmienione} \\ \hline
25 mar 2015          & Będźkowski Piotr     & Utworzenie dokumentu.         \\ \hline
26 mar 2015          & Będźkowski Piotr     & Modyfikacja treści.           \\ \hline
14 cze 2015          & Będźkowski Piotr     & Modyfikacja treści.           \\ \hline
\end{tabularx}
