\section{Cel dokumentu}
\suppressfloats[t]

Dokument ma na celu definicję
wszystkich wymagań, jakie ma spełniać system.

\section{Ostateczny opis wymagań}

\subsection{Wstęp}
System ma na celu ułatwienie i zautomatyzowanie procesu organizacji
dorocznych konferencji instytutu. Podczas takiego wydarzenia pracownicy 
wraz z członkami rodzin wyjeżdżają na kilka dni. Pracownicy w tym czasie
uczestniczą w konferencjach przez nich przygotowanych prezentując własne
artykuły wcześniej zrecenzowane przez osoby spoza instytutu. Rodziny 
uczestniczą w wycieczkach. \par
W przygotowaniach do konferencji będą uczestniczyli pracownicy,
oraz członkowie ich rodzin wykorzystując System Obsługi Konferencji. Każda
zainteresowana osoba będzie miała do niego dostęp i będzie brała udział
w interesujących ją procesach przygotowujących konferencję.
\subsection{Definicja wymagań funkcjonalnych}
\subsubsection{Administrator}
Administrator, jako osoba mająca absolutną kontrolę nad systemem, nad którą
spoczywa odpowiedzialność za całościowe przygotowanie konferencji może 
zarządzać innymi użytkownikami i konfiguracją systemu. \par
Jedną z podstawowych ról administratora jest umożliwienie dostępu do
Systemu pozostałym zainteresowanym. W tym celu Administrator ma możliwość 
utworzyć konto odpowiedniej osobie naj jej wniosek lub z własnej 
inicjatywy. Wyróżniamy 4 rodzaje kont użytkowników:
\begin{description}
    \item[Konto Administratora] \hfill \\
        konto z pełnym dostępem do wszystkich funkcji systemu.
        Do poprawnego działania aplikacji musi być aktywne co
        najmniej jedno konto Administratora. To on przydziela uprawnienia,
        zadania, role i dostępy pozostałym członkom.

    \item[Konto Pracownika] \hfill \\
        Każdy pracownik instytutu powinien posiadać to konto, 
        o ile chce brać aktywny udział w Konferencji. Podstawowymi 
        własnościami takiego konta, jest możliwość zgłoszenia swojego 
        artykułu do recenzji i późniejszego poprowadzenia prelekcji.
        Do konta pracowniczego dołączane są konta członków rodziny 
        pracownika.

    \item[Konto Członka rodziny] \hfill \\
        Może należeć do jednej lub większej ilości osób. Musi być 
        przypisane do dokładnie jednego pracownika. Nie daje dostępu do
        naukowo-dydaktycznej części systemu.

    \item[Dostęp Recenzenta] \hfill \\
        Nie jest kontem w pełnym tego słowa znaczeniu, ponieważ
        ma za zadanie dać możliwość uczestnictwa recenzentów w procesie
        oceniania artykułu. System powinien działać w sposób przezroczysty
        dla recenzenta, jednocześnie ukrywając tożsamość autora artykułu
        i recenzenta przed nimi samymi.

\end{description}
Administrator powinien mieć też możliwość wyświetlenia listy wszystkich 
kont w systemie i ich uprawnień. Ponadto dla każde wyświetlane konto
może edytować. Oznacza to, że może edytować dane opisujące to konto,
ustawić nowe hasło dla użytkownika, a także usunąć konto, lub je 
zamknąć. \par
Administrator w procesie oceny artykuły ma możliwość oddelegować 
zarządzanie wycieczkami do innego użytkownika, oraz zaakceptować
program poszczególnej wycieczki. \par
W procesie oceny artykułu, administrator może ocenić postęp zaawansowania
postępów oceny, może zaingerować w sam proces.\par
\subsubsection{Ocena artykułu}
Artykuł jest dokumentem napisanym przez pracownika, na temat którego
w czasie konferencji ma zostać wygłoszona prelekcja. Powinien on zostać
wcześniej zrecenzowany przez osobę zaznajomioną z tematem, o którym
artykuł traktuje. Jednocześnie recenzent nie powinien znać tożsamości
autora artykułu i vice versa. Z tego powodu Recenzentami są najczęściej
osoby spoza instytutu. Proces recenzji artykułu polega na naprzemiennej
wymianie uwag i sugestii pomiędzy recenzentem i autorem, nanoszeniu
poprawek do momentu zaakceptowania artykułu przez recenzenta. 
\subsubsection{Pracownik}
Konto pracownicze daje dostęp do zasobów systemu ukierunkowanych na
przygotowanie części merytorycznej konferencji. Każdy pracownik może 
zgłosić swój artykuł, głosować na inne, oraz mieć podgląd na ranking 
najwyżej ocenionych artykułów, które zostały akceptowane. \par
Pracownik, jako osoba zainteresowana dyskusjami dydaktycznymi może 
zgłaszać propozycje takich tematów, komentować inne propozycje.
W kwestii wycieczek i atrakcji turystycznych Pracownik ma identyczne 
uprawnienia jak Członek Rodziny.
\subsubsection{Członek Rodziny}
Członek Rodziny jest osobą towarzyszącą Pracownikowi podczas Konferencji.
Pojedyncze konto Członka Rodziny może należeć jednocześnie do jednej lub
kilku osób (np. matki i małego dziecka). Członek Rodziny ma pełen dostęp
do części systemu odpowiedzialnej za organizację wycieczek i innych
atrakcji poza naukowych. Może on wyświetlić listę wycieczek, zaproponować
nową, wyświetlić jej szczegóły, zapisać się na atrakcję turystyczną,
komentować. Dodatkowo może poznać ofertę gastronomiczną miejsc odwiedzanych
podczas konferencji, wyświetlić harmonogram całej imprezy,
oraz odzyskać zapomniane lub utracone hasło.

\section{Opis ewolucji systemu}
Kolejnym etapem życia systemu po przeanalizowaniu wymagań systemu, jest
jego implementacja. System Obsługi Konferencji ma służyć zakładowi przez
wiele lat i umożliwiać rozszerzanie jego funkcjonalności zależnie od 
potrzeb.


\section{Historia Zmian}

\begin{tabularx}{\textwidth}{X|l|X}
\hline
\textbf{Data zmiany} & \textbf{Kto zmienił} & \textbf{Co zostało zmienione} \\ \hline
16 czerwca 2015          & Wojciech Decker & Utworzenie dokumentu.         \\ \hline
\end{tabularx}
