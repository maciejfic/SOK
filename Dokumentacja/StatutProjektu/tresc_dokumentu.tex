\section{Cel dokumentu}
\suppressfloats[t]

Dokument ma na celu przedstawienie uzasadnienia powstania Systemu Obsługi Konferencji (SOK). Dokuemnt także krótko omawia założenia, cele projektu oraz oczekiwane rezultaty.


\section{Uzasadnienie projektu}

Powstanie projektu jest odpowiedzią na potrzeby pracowników Instytutu Sterowania i Elektroniki Przemysłowej Politechniki Warszawskiej.
Każdego roku organizowana jest konferencja instytutu mająca na celu podsumowanie osiągnieć dydaktycznych i naukowych oraz wyznaczanie nowych kierunków rozwoju. 

\bigskip
\noindent Dotychczasowy sposób organizacji od strony techniczno/formalnej ze względu na zmiany w sposobie pracy instytutu przestał się sprawdzać. 

\bigskip
\noindent W związku z tym postanowniono, że do celów organizacji konferencji powstanie dedykowany system informatyczny, który umożliwi dalsze funkcjonowanie w sprawny i przejrzysty sposób. 

\section{Cele}

Celem projektu jest usprawnienie organizacji konferencji i wprowadzenie metody formalnej komunikacji pomiędzy uczestnikami procesu organizacji konferencji i jej uczestnikami. 

\section{Oczekiwane rezultaty}

Po udanym wdrożeniu projektu oczekujemy przede wszystkim pozytywnej reakcji od wszystkich stron uczestniczących w konferencjach i aktywnego korzystania z systemu. Przez pozytywną reakcję rozumiemy, że wszystkie osoby, które przewidziane są jako użytkownicy nie będą odrzucali wprowadzonego rozwiązania i porzucą obecnie używane kanały komunikacji/organizacji na rzecz powstałego systemu.

\section{Założenia i ograniczenia}

Powstawanie projektu podzielone jest na dwie fazy, wykonanie dokumentacji zamówienia oraz późniejsze wykonanie.

\bigskip
\noindent Wykonaniem dokumentacji zajmuje się oddzielny zespół projektowy.

\section{Zestawienie wykonawców}

Zespół projektowy wykonujący dokumentację składa się z czterech osób. W skład zespołu wchodzą: Maciej Fic, Wojciech Decker, Piotr Błądek i Piotr Będźkowski.

\bigskip
\noindent Skład osobowy zespołu wykonującego aplikację nie został zdefiniowany. Jest to przewidziane jako dalszy etap projektu.

\section{Historia Zmian}

\begin{tabularx}{\textwidth}{X|l|X}
\hline
\textbf{Data zmiany} & \textbf{Kto zmienił} & \textbf{Co zostało zmienione} \\ \hline
02 cze 2015          & Będźkowski Piotr     & Utworzenie dokumentu.           \\ \hline
13 cze 2015          & Będźkowski Piotr     & Modyfikacja treści.           \\ \hline
\end{tabularx}
